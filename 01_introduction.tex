\chapter{Introduction-TEST}
\label{chapter:introduction}

\section{Why?}

Although the internet provides plenty of Python Code examples for almost all imaginable questions, these examples are not always accessible to people who are new to writing code.

Frequently, there are multiple ways of coding a solution for the same program. Access to multiple options can be confusing to newcomers as they may not have the knowledge to be able to discern which options are better than others.

Experienced coders have an understanding of the kinds of structures and algorithms they want to use, and thus know what to search for when learning a new programming language. Those who are new to coding, do not necessarily know what to search for.

\section{Who?}

Hence, this resource is for newcomers to coding with Python. There is no expectation that the reader is familiar with programming concepts. Those who are coming from another language may still find the resource helpful to identify common patterns of use.

\section{What?}

The following chapters provide an opinioned collation of exemplar snippets of code which are commonly used in data analytics. Selection of exemplars is based on (a) relevance to the data analytics topic, (b) commonly used by analysts working with Python, and (c) easy to understand by newcomers.

Satisfying these constraints may mean that the exemplars are not always the most efficient, nor the most compact code.

\section{How?}

Each chapter is arranged from easier exemplars to more complex exemplars, and so the reader is encouraged not to read from front to back, but to dip into the appropriate chapter/s to find exemplar/s that are helpful to the particular task they are working on at the time.

\section{Where?}

Exemplars are provided in 6 sections. Chapter 2 provides some basics that are used across the board and that provide a helpful foundation for more specific exemplars. Chapters 3 - 7 cover 5 topics of data analytics: (3) Reading data from external sources, (4) Cleaning data, (5) Analysing data, (6) Visualising data, and (7) Sharing data.

\section{When?}

Use the resource as you need it. It is designed to be a reference rather than a tutorial.