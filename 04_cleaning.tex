\chapter{Cleaning}
\label{chapter:cleaning}

Data are rarely without problems. Aspects of the data can be missing, or incomplete, or even false. Data analytics can involve a significant amount of time dealing with inconsistencies and anomalies in the original data prior to starting the main analysis. This process of transforming \textit{dirty} data into something usable is known as \textbf{cleaning}. The following examples show some common code that is useful in the cleaning process.

\section{Not a Number (NaN)}

When a dataframe expects a value to be present, but it is missing (\textit{null}), a \code{NaN} fills the cell in the place of the expected value. A helpful cleaning process is to identify which columns include NaNs and replace them with a meaningful value.

\textbf{Important:} a meaningful replacement value is dependent on the meaning of the data in the column. For example, if a column includes measured temperatures from a sensor, a \code{NaN} may mean that the sensor is offline. In this case, replacing the value with a \code{0} would impact the truth of the data (unless the temperature was actually 0 at the time).

To see the number of NaNs in each column in the dataframe the \code{isna()} function can be used together with \code{sum()}.

\begin{pycode}
    # display the number of NaNs in each column of the data
    print(df.isna().sum())
\end{pycode}

To replace all NaNs in a particular column with a given value, use \code{fillna()}.

\begin{pycode}
    # replace NaNs in col1 with string 'OTHER'
    column = 'col1'
    replace_value = 'OTHER'
    df[column] = df[column].fillna(replace_value)
\end{pycode}

The same code will work for replacing multiple columns. However, take care that the replacement data means the same thing for each of the columns.

\begin{pycode}
    # replace NaNs in multiple columns with the string 'N/A'
    columns = ['col1','col5']
    replace_value = 'N/A'
    df[columns] = df[columns].fillna(replace_value)
\end{pycode}



\section{Whitespace}

Sometimes cells look empty, but they actually contain invisible characters (like a space: \code{' '}). To see whether a column includes \textit{whitespace}, turn the column into a list and print the list. Note, this may not be a good idea if you have a lot of rows in your dataframe!

\begin{pycode}
    # column as a list
    column_name = 'col1'
    print(list(df[column_name]))
\end{pycode}

To replace all whitespace, use the \code{replace()} function of the dataframe. By using \textbf{Regex} (regular expressions), it is possible to replace empty strings as well as strings with whitespace characters. The regex pattern is \code{r'^\s*$'} and it requires that the python regex library has been imported.

\begin{pycode}
    # import the regex library
    import re
    # Replace whitespace major column
    column_name = 'col1'
    replace_value = '_empty_'
    df[column_name] = df[column_name].replace(r'^\s*$', replace_value, regex=True)
\end{pycode}