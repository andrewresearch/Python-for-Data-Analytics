\chapter{Reading}
\label{chapter:reading}

\section{Structured data and dataframes}

When working with structured data, one of the most common file types is the *CSV* (comma separated value) file. CSV files are plain text files that are structured such that every line of the file represents a row of data, and every column in the data is separated by a comma. Thus the data is structured like:

\begin{verbatim}
    col1,col2,col3\n
    r1c1,r1c2,r1c3\n
    r2c1,r2,c2,r2c3
\end{verbatim}

The \code{pandas} library provides a function for reading CSV files directly into a dataframe.
Each of the following examples requires the library, so the following \code{import} needs to be included prior to any of these examples.

\begin{pycode}
    import pandas as pd
\end{pycode}

The simplest \code{pandas.read_csv()} function assumes a header row and no index. After reading a CSV into a dataframe, it is a good idea to check that the shape (rows x columns) of the original  file is reflected in the shape of the dataframe. The \code{shape} property of the dataframe will output a tuple of the total number of rows and columns (rows, cols).

\begin{pycode}
    # Read a CSV into a dataframe
    file_path = "data/"
    file_name = "my_data.csv"
    df = pd.read_csv(f"{file_path}{file_name}")
    print(df.shape) # displays (rows,cols)
\end{pycode}

If the first column of the data can be used as an index (each row is a unique value), then the column name (or number) can be passed.

\begin{pycode}
    # Read a CSV into a dataframe with an index on idx
    file_path = "data/"
    file_name = "my_data.csv"
    df = pd.read_csv(f"{file_path}{file_name}",index_col='idx')
    print(df.shape) # displays (rows,cols)
\end{pycode}

\newpage
If there is no header row in the data, then one can be supplied.

\begin{pycode}
    # Read a CSV without a header into a dataframe and supply column names
    file_path = "data/"
    file_name = "my_data.csv"
    colnames = ['col1','col2','col3']
    df = pd.read_csv(f"{file_path}{file_name}",names=colnames)
    print(df.shape) # displays (rows,cols)
\end{pycode}

