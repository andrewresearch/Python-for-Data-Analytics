\chapter{Reading}
\label{chapter:reading}

\section{Structured data and dataframes}

When working with structed data, one of the most common file types is the *CSV* (comma separated value) file. CSV files are plain text files that are structured such that every line of the file represents a row of data, and every column in the data is separated by a comma. Thus the data is structured like:

\begin{verbatim}
    col1,col2,col3\n
    r1c1,r1c2,r1c3\n
    r2c1,r2,c2,r2c3
\end{verbatim}

\subsection{Reading CSV data from a file}

The \code{pandas} library provides a function for reading CSV files directly into a dataframe.
Each of the following examples requires the library, so the following \code{import} needs to be included prior to any of these examples.

\begin{pycode}
    import pandas as pd
\end{pycode}

The simplest \code{pandas.read_csv()} function assumes a header row and no index. After reading a CSV into a dataframe, it is a good idea to check that the shape (rows x colums) of the original  file is reflected in the shape of the dataframe. The \code{shape} property of the dataframe will output a tuple of the total number of rows and columns (rows, cols).

\begin{pycode}
    # Read a CSV into a dataframe
    file_path = "data/"
    file_name = "my_data.csv"
    df = pd.read_csv(f"{file_path}{file_name}")
    print(df.shape) # displays (rows,cols)
\end{pycode}

If the first column of the data can be used as an index (each row is unique value), then the column name (or number) can be passed.

\begin{pycode}
    # Read a CSV into a dataframe with an index on idx
    file_path = "data/"
    file_name = "my_data.csv"
    df = pd.read_csv(f"{file_path}{file_name}",index_col='idx')
    print(df.shape) # displays (rows,cols)
\end{pycode}

\newpage
If there is no header row in the data, then one can be supplied with the \code{name} parameter.

\begin{pycode}
    # Read a CSV without header into a dataframe and supply column names
    file_path = "data/"
    file_name = "my_data.csv"
    colnames = ['col1','col2','col3']
    df = pd.read_csv(f"{file_path}{file_name}",names=colnames)
    print(df.shape) # displays (rows,cols)
\end{pycode}


\subsection{Reading CSV data from a URL}

CSV data can be read into a pandas dataframe in the same way as from a file. To check if a server supports this, you should be able to visit the URL in a web browser and see the CSV structured text.

\begin{pycode}
    # Read CSV data from a URL
    url =  "https://my.data.source.api/url_endpoint"
    df = pd.read_csv(url)
\end{pycode}

The same optional parameters (e.g. \code{index_col}, \code{names}) that are used with files will also work with a valid URL.

\subsection{Reading Excel data}

The pandas library provides additional functions for reading other structured data, including the \code{read_excel()} function. This facilitates opening Microsoft Excel files as dataframes directly without converting them to CSV format. Files should be in the newer \code{.xlsx} format.

\begin{pycode}
    # Read an Excel file
    file_path = "data/"
    excel_file_name = "my_excel_data.xlsx"
    df = pd.read_excel(f"{file_path}{excel_file_name}")
\end{pycode}

If the excel file has multiple sheets, the specific sheet to read in can be specified with the \code{sheet_name} parameter.

\begin{pycode}
    # Read an Excel file
    file_path = "data/"
    excel_file_name = "my_excel_data.xlsx"
    sheet = "my_sheet"
    sheet_df = pd.read_excel(f"{file_path}{excel_file_name}",sheet_name=sheet)
\end{pycode}

\newpage
\section{Semi-structured data and JSON}

\subsection{Reading plain text files into a string}

Plain text files can be read directly into a variable. After doing so, the variable contents will be a \code{string} of characters that are the same as the text file (including invisible characters like spaces and end of line characters).

\begin{pycode}
    # Read in a plain text file
    file_path = "data/"
    file_name = "my_plain_text.txt"
    with open(f"{file_path}{file_name}", 'r') as file:
    text = file.read()

    print(text)
\end{pycode}

\subsection{Reading plain text files into a list of strings}

Sometimes data saved in a text file is written in a way that each \code{line} of the file corresponds to a \textit{record} in the data (e.g. CSV files, or log files). Typically an end of line character like a new line \code{\n} or a carriage return \code{\r} is used to denote the end of the line. Unix and MacOS systems tend to use \code{\n} while Microsoft systems often use both \code{\n\r}.

Python can recognise these line ending characters and read files line by line into a list. In this case, the \code{readlines()} function reads the file into a variable that is a \code{list} of \code{string} where each \code{string} is a line in the file.

\begin{pycode}
    # Read in a plain text file
    file_path = "data/"
    file_name = "my_plain_text.txt"
    with open(f"{file_path}{file_name}", 'r') as file:
    lines_list = file.readlines()

    # Display the list
    print(lines_list)

    # Iterate over the list and print each line of the list
    for line in lines_list:
    print(line)
\end{pycode}

If both the full text and a list of lines is required, the data can be read using \code{read()} function, and then the string can be split on the end of line characters using the \code{split()} function.


\begin{pycode}
    # Read in a plain text file and split into list of lines
    file_path = "data/"
    file_name = "my_plain_text.txt"
    with open(f"{file_path}{file_name}", 'r') as file:
    full_text = file.read()

    # Display the string (full_text)
    print(full_text)

    # Split the string into lines
    lines_list = full_text.split('\n')  # <-- may need to be '\r\n'

    # Dispaly the  list of strings
    print(lines_list)

\end{pycode}

\subsection{Reading text files in JSON format into a dict}

The \code{JSON} format is a common file format for \textit{semi-structured} data. The format is essentially a combination of \code{key:value} pairs and lists. The data is structured like:

\begin{verbatim}
    {"key1":"value1",
     "key2_list": ["item1","item2","item3"],
     "key4_num": 2023,
     "key5":[45,54,63,72]}
\end{verbatim}

This structured can be represented in Python with a combination of \code{dict} and \code{list} types.

A JSON file can be read into python by first reading the text into a \code{string} variable, and then converting the string into a python \code{dict} using the \code{loads()} function from the \code{json} library. The library needs to be imported first.

\begin{pycode}
    import json

    # Read in a plain text file
    file_path = "data/"
    file_json = "my_json_file.json"
    with open(f"{file_path}{file_json}", 'r') as file:
    json_string = file.read()

    # convert the json string
    my_dict = json.loads(json_string)

    # display the dict
    print(my_dict)
\end{pycode}

If the original string is not required, the dict can be created within the \code{with open()} command.

\begin{pycode}
    # Read in a plain text file
    file_path = "data/"
    file_json = "my_json_file.json"
    with open(f"{file_path}{file_json}", 'r') as file:
    my_dict = json.loads(file)

    # display the dict
    print(my_dict)
\end{pycode}

TIP: See section \ref{chapter:analysing} for exemplars on how to access the data within a dict.

