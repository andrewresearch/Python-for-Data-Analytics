\chapter{Analysing}
\label{chapter:analysing}

\section{Counting categorical data}

\begin{pycode}
    import pandas as pd
    data = {"pet":['dog','cat','cat','dog','dog','dog'],
            "colour":["black","brown","black","brown","black","brown"],
            "count":[3,1,4,2,1,2]}
    df = pd.DataFrame(data)
    df
\end{pycode}

\begin{tabular}{l l l l}
     & pet  &  colour & count \\
    \hline
    0&	dog&	    black	 &   3\\
    1&	cat&	    brown	 &   1\\
    2&	cat&	    black	 &   4\\
    3&	dog&	    brown	 &   2\\
    4&	dog&	    black	 &   1\\
    5&	dog&	    brown	  &  2\\
\end{tabular}

Count the number of `dog' and `cat' entries in the `pet' column:

\begin{pycode}
    result = df.groupby(['pet']).size()
    print(result)
\end{pycode}

\begin{verbatim}
    pet
    cat    2
    dog    4
    dtype: int64
\end{verbatim}

Get a total number of dogs and cats by using \code{sum()} on the `count' column:

\begin{pycode}
    result = df.groupby(['pet'])['count'].sum()
    print(result)
\end{pycode}

\begin{verbatim}
    pet
    cat    5
    dog    8
    Name: count, dtype: int64
\end{verbatim}


\newpage

Get totals based on categories and subcategories:

\begin{pycode}
    result = df.groupby(['pet','colour']).sum()
    print(result)
\end{pycode}

\begin{verbatim}
    pet colour  count
    cat black       4
        brown       1
    dog black       4
        brown       4
\end{verbatim}

Get a total number of coloured pets and format them as a dataframe:

\begin{pycode}
    new_df = df.groupby(['colour'])['count'].sum().reset_index(name="count")
    new_df
\end{pycode}

\begin{tabular}{l l l}
    &colour	&count \\
    \hline
    0	&black	&8 \\
    1	&brown	&5\\
\end{tabular}